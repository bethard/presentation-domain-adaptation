% !TEX TS-program = pdflatexmk

\documentclass[14pt]{beamer}
\usepackage{newtxtext,newtxmath}
\usepackage{microtype}
\usepackage[english]{babel}

% Define UA colors
% https://brand.arizona.edu/guide/color
\definecolor{ua-red}{HTML}{AB0520}
\definecolor{ua-blue}{HTML}{0C234B}

\mode<presentation>{
\usetheme{Madrid}
\usecolortheme[named=ua-red]{structure}
\setbeamertemplate{navigation symbols}{}
\setbeamertemplate{footline}[frame number]
\setbeamertemplate{section in toc}[square]
\setbeamertemplate{subsection in toc}[square]
\setbeamertemplate{items}[square]
\setbeamercovered{transparent=0}
}

\author[Bethard]{Dr. Steven Bethard}
\institute[Arizona]{%
School of Information\\
University of Arizona}

\AtBeginSection[]
{
  \begin{frame}<beamer>{Outline}
    \tableofcontents[currentsection]
  \end{frame}
}

\title{Adapting natural language processing models across clinical domains}
\date[]{26 Feb 2021}


\begin{document}


\begin{frame}
  \titlepage
\end{frame}

\section{Domain adaptation is critical for clinical NLP}

\begin{frame}{Machine learning needs domain adaptation}
% example of parsing times and punctuation?
% followed by table of performance drops
\end{frame}

\begin{frame}{Domain adaptation is hard}
% new phrases
% new distribution of labels for old phrases
\end{frame}

\begin{frame}{Clinical domain adaptation is even harder}
% differences across institutions
% data sharing constraints
% etc.
\end{frame}

\section{Domain adaptation comes in many forms}

\begin{frame}{Taxonomy of domain adaptation}
% from JAMIA Open
\end{frame}

\begin{frame}{Frustratingly easy domain adaptation}
% diagram of method
% why it doesn't work for Clinical NLP
\end{frame}

\begin{frame}{Self-training}
% maybe the tri-training paper?
% diagram of method
% why it doesn't work for Clinical NLP
\end{frame}

\section{Domain adaptation for clinical information extraction}

\begin{frame}{Source-free domain adaptation}
% summarize shared tasks: negation and time
\end{frame}

\begin{frame}{Source-free self-training}
% Xin's results
\end{frame}

\begin{frame}{Online active learning}
% any results from Xin
\end{frame}

\section{Domain adaptation for clinical concept normalization}

\begin{frame}{Clinical concept normalization}
\end{frame}

\begin{frame}{Normalization as vector-space search}
\end{frame}

\begin{frame}{Domain adaptation without retraining}
\end{frame}

\section*{Summary and future work}

\begin{frame}{Summary}
\end{frame}

\begin{frame}{Future work}
\end{frame}


\end{document}
